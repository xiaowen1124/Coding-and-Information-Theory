\section{联合熵与条件熵}
\subsection{联合熵}
\begin{definition}
设一维随机变量$(\xi, \eta )$的联合分布为$$p(x,y)=P_r\{\xi=x, \eta=y\}, x \in \mathscr{X}, y \in \mathscr{Y} $$

    对于 $ x \in \mathscr{X}, y \in \mathscr{Y} $, 二维随机变量 $ \xi, \eta $ 的联合熵定义为:
$$
H(\xi, \eta)=-\sum_{x \in \mathscr{X}} \sum_{y \in \mathscr{Y}} p(x, y) \log _{2} p(x, y)
$$
或者写成数学期望的形式:
$$H(\xi, \eta)=-E \left( \log_2 p(x,y)\right)$$
\end{definition}

\begin{remark}
    若 $ \xi, \eta $ 是相互独立的, 即 $ p(x, y)=p(x) p(y) $, 则二维随机变量 $ \xi, \eta $ 的联合熵满足 $ H(\xi, \eta)=H(\xi)+H(\eta) $
\end{remark}
\begin{proof}

$$
\begin{aligned}
 H(\xi, \eta)&=-\sum_{x \in \mathscr{X}} \sum_{y \in \mathscr{Y}} p(x, y) \log p(x, y) \\
&=-\sum_{x \in \mathscr{X}} \sum_{y \in \mathscr{Y}} p(x, y)[\log p(x)+\log p(y)] \\
&=-\sum_{x \in \mathscr{X}} \sum_{y \in \mathscr{Y}} p(x, y) \log p(x)-\sum_{x \in \mathscr{X}} \sum_{y \in \mathscr{Y}} p(x, y) \log p(y) \\
&=-\sum_{x \in \mathscr{X}}\left(\sum_{y \in \mathscr{Y}} p(x, y)\right) \log p(x)-\sum_{y \in \mathscr{Y}}\left(\sum_{x \in \mathscr{X}} p(x, y)\right) \log p(y) \\
&=-\sum_{x \in \mathscr{X}} p(x) \log p(x)-\sum_{y \in \mathscr{Y}} p(y) \log p(y) \\
&=H(\xi)+H(\eta)
\end{aligned}
$$
\end{proof}
\begin{remark}
     $ -\sum\limits_{x \in \mathscr{X}} \sum\limits_{y \in \mathscr{Y}} p(x, y) \log p(x), x $ 取定, $ y \in \mathscr{Y} $, 求和与 $ x $ 无关, 故 $$-\sum_{x \in \mathscr{X}}\left(\sum_{y \in \mathscr{Y}} p(x, y)\right) \log p(x)=-\sum_{x \in \mathscr{X}} p(x) \log p(x) $$
\end{remark}


\subsection{条件熵}
设随机变量对 $ (\xi, \eta) $ 有联合分布 $ p(x, y) $, 用
$$
p(y \mid x)=P_{r}\{\eta=y \mid \xi=x\}, x \in \mathscr{X}, y \in \mathscr{Y}
$$
表示条件概率分布, 则给定 $ \xi=x $ 条件下 $ \eta $ 的熵定义为
$$
H(\eta\mid \xi=x)=-\sum_{y \in \mathscr{Y}} p(y \mid x) \log p(y \mid x)
$$
而给定随机变量 $ \xi $ 条件下 $ \eta $ 的熵记为 $ H(\eta \mid \xi) $, 它是 $ H(\eta \mid \xi=x) $ 关于 $ \xi $ 的平均值,即


\begin{definition}
    如果 $ (\xi, \eta) \sim p(x, y) $, 那么 $ \eta $ 关于 $ \xi $ 的条件熵定义为:
$$
\begin{aligned}
H(\eta \mid \xi) & =\sum_{x \in \mathscr{X}} p(x) H(\eta \mid \xi=x) \\
& =-\sum_{x \in \mathscr{X}} p(x) \sum_{y \in \mathscr{Y}} p(y \mid x) \log p(y \mid x) \\
& =-\sum_{x \in \mathscr{X}} \sum_{y \in \mathscr{Y}} p(x, y) \log p(y \mid x)\\
&=-E(\log p(\eta \mid \xi))
\end{aligned}
$$

同样地, 有 $ H(\xi \mid \eta)=-\sum\limits_{x \in \mathscr{X}} \sum\limits_{y \in \mathscr{Y}} p(x, y) \log p(x \mid y) $
\end{definition}

\subsection{联合熵和条件熵的关系}
\begin{theorem}
     联合熵与条件熵的关系为:
$$
H(\xi, \eta)=H(\xi)+H(\eta \mid \xi)=H(\eta)+H(\xi \mid \eta)=H(\eta, \xi)
$$
\end{theorem}
\begin{proof}
    
\end{proof}
$$
\begin{aligned}
H(\xi, \eta)&=-\sum_{x \in \mathscr{X}} \sum_{y \in \mathscr{Y}} p(x, y) \log p(x, y)\\
&=-\sum_{x \in \mathscr{X}} \sum_{y \in \mathscr{Y}} p(x, y) \log p(x) p(y \mid x) \\
&=-\sum_{x \in \mathscr{X}} \sum_{y \in \mathscr{Y}} p(x, y)[\log p(x)+\log p(y \mid x)] \\
&=-\sum_{x \in \mathscr{X}} \sum_{y \in \mathscr{Y}} p(x, y) \log p(x)-\sum_{x \in \mathscr{X}} \sum_{y \in \mathscr{Y}} p(x, y) \log p(y \mid x) \\
&=-\sum_{x \in \mathscr{X}}\left(\sum_{y \in \mathscr{Y}} p(x, y)\right) \log p(x)-\sum_{x \in \mathscr{X}} \sum_{y \in \mathscr{Y}} p(x, y) \log p(y \mid x) \\
&=-\sum_{x \in \mathscr{X}} p(x) \log p(x)-\sum_{x \in \mathscr{X}} \sum_{y \in \mathscr{Y}} p(x, y) \log p(y \mid x) \\
&=H(\xi)+H(\eta \mid \xi)
\end{aligned}
$$
同理可证明 $ H(\xi, \eta)=H(\eta)+H(\xi \mid \eta) $.
根据此关系式可由联合熵和随机变量的熵求出条件熵.

\subsection{联合熵和条件熵的性质}

\textbf{性质1:} $ H(\eta \mid \xi) \geqslant 0 $, 等号成立的充要条件为 $ \eta $ 是由 $ \xi $ 决定的随机变量.

\begin{remark}
    若$ p(y \mid x)=1 $, 则在 $ \xi $ 的条件下, $ \eta $ 一定发生.
\end{remark}


\textbf{性质2:} $ H(\xi, \eta) \geqslant H(\xi) $ (或者 $ H(\xi, \eta) \geqslant H(\eta) $ ),等号成立的充要条件为 $ \eta $ 是由 $ \xi $ 决定的随机变量(或者 $ \xi $ 是由 $ \eta $ 决定的随机变量).

\textbf{性质3:} $ H(\xi \mid \eta)=H(\xi) $ 的充要条件为 $ \xi, \eta $ 是相互独立的.
$$
\begin{array}{c}
p(x, y)=p(x) p(y), p(x \mid y)=\frac{p(x) p(y)}{p(y)}=p(x) \\
\left(p(x \mid y)=p(x), \quad y \text { 不影响 } x \right)
\end{array}
$$

\begin{example}
令 $ (\xi, \eta) $ 具有如下联合分布
\begin{table}[ht]
\centering 
\begin{tabular}{|c|cccc|c|}
\hline
\diagbox{$ \eta $}{$ \xi $}
 & $1$ & $2$ & $3$ & $4$ & $ \sum $ \\
\hline $1$ & $ \frac{1}{8} $ & $ \frac{1}{16} $ & $ \frac{1}{32} $ & $ \frac{1}{32} $ & $ \frac{1}{4} $ \\
$2$ & $ \frac{1}{16} $ & $ \frac{1}{8} $ & $ \frac{1}{32} $ & $ \frac{1}{32} $ & $ \frac{1}{4} $ \\
$3$ & $ \frac{1}{16} $ & $ \frac{1}{16} $ & $ \frac{1}{16} $ & $ \frac{1}{16} $ & $ \frac{1}{4} $ \\
$4$ & $ \frac{1}{4} $ & $0$ & $0$ & $0$ & $ \frac{1}{4} $ \\
\hline $\sum$ & $\frac{1}{2}$ & $\frac{1}{4}$ & $\frac{1}{8}$ & $\frac{1}{8}$ & $1 $ \\
\hline
\end{tabular}
\end{table}

$ \xi, \eta $ 的边际分布分别为:
$$
\xi \sim\left(\begin{array}{cccc}
1 & 2 & 3 & 4 \\
\frac{1}{2} & \frac{1}{4} & \frac{1}{8} & \frac{1}{8}
\end{array}\right) \quad \eta \sim\left(\begin{array}{cccc}
1 & 2 & 3 & 4 \\
\frac{1}{4} & \frac{1}{4} & \frac{1}{4} & \frac{1}{4}
\end{array}\right)
$$
对 $ \xi: p(x)=\sum\limits_{y \in \mathscr{Y}} p(x, y) $ ( $ x $ 定).

故 $ p(1)=p(1,1)+p(1,2)+p(1,3)+p(1,4)=\frac{1}{8}+\frac{1}{16}+\frac{1}{16}+\frac{1}{4}=\frac{1}{2} $,

$ p(2)=p(2,1)+p(2,2)+p(2,3)+p(2,4)=\frac{1}{16}+\frac{1}{8}+\frac{1}{16}+0=\frac{1}{4} $

于是, 可求得 
$$ H(\xi)=\sum_{i=1}^{4} p_{i} \log \frac{1}{p_{i}}=\frac{1}{2} \log _{2} 2+\frac{1}{4} \log _{2} 4+2 \frac{1}{8} \log _{2} 8=\frac{1}{2}+\frac{2}{4}+\frac{3}{4}=\frac{7}{4} $$

$$H(\eta)=\sum_{i=1}^{4} p_{i} \log \frac{1}{p_{i}}=4 \cdot \frac{1}{4} \log _{2} 4=4 \cdot \frac{1}{2}=2$$

$$
\begin{aligned}
H(\xi, \eta)&=\sum_{x \in \mathscr{X}} \sum_{y \in \mathscr{Y}} p(x, y) \log \frac{1}{p(x, y)} \\
&=2 \cdot \frac{1}{8} \log 8+6 \cdot \frac{1}{16} \log 16+4 \cdot \frac{1}{32} \log 32+\frac{1}{4} \log 4 \\
&=\frac{3}{4}+\frac{6}{4}+\frac{5}{8}+\frac{1}{2}=\frac{27}{8} \\
H(\xi \mid \eta)&=H(\xi, \eta)-H(\eta)=\frac{27}{8}-2=\frac{11}{8} \\
H(\eta \mid \xi)&=H(\xi, \eta)-H(\xi)=\frac{27}{8}-\frac{7}{4}=\frac{13}{8}
\end{aligned}
$$

\end{example}