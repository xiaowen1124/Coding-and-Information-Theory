
\section{凸函数及其应用}
\subsection{凸函数的定义与判别}
\begin{definition}
     设 $ g(x) $ 是定义在区间 $ (a, b) $ 上的函数,如果对任意的 $ x_{1}, x_{2} \in(a, b) $ 和任意的 $ 0 \leqslant \lambda \leqslant 1 $, 都有
$$
g\left(\lambda x_{1}+(1-\lambda) x_{2}\right) \geqslant \lambda g\left(x_{1}\right)+(1-\lambda) g\left(x_{2}\right)
$$
则称 $ g(x) $ 是 $ (a, b) $ 上的上凸函数,如果等号只有当 $ \lambda=0 $ 或 $ \lambda=1 $或 $ x_{1}=x_{2} $ 时才成立,则称函数 $ g $ 是严格上凸的(若不等式相反称是下凸的).
\end{definition}
\begin{remark}

    (1)如果 $ g $ 是上凸(严格上凸)的,那么 $ -g $ 必是下凸(严格下凸)的;
    
(2) 函数是上凸的, 那么它的函数值总是位于任意弦的上方. 函数是下凸的, 那么它的函数值总是位于任意弦的下面.如$x^2,|x|,e^x,x\log x$等是下凸函数,$\sqrt{x}, \log x$是上凸函数. 注意$ax+b$是上凸也是下凸的.

(3) 像许多信息量,如熵和互信息都具有上凸性.
\end{remark}
\begin{theorem}[判定定理]
    如果函数 $ g $ 在任意处都有非负(正)二阶导数,则函数 $ g $ 是下凸(严格下凸的)
\end{theorem}
\begin{proof}
    函数 $ g $ 在的处的泰勒展开式
$$
g(x)=g\left(x_{0}\right)+g^{\prime}\left(x_{0}\right)\left(x - x_{0}\right)+\frac{1}{2} g^{\prime \prime}\left(x_{*}\right)\left(x-x_{0}\right)^{2}
$$
其中 $ x_*$ 在 $ x $ 与 $ x_{0} $ 之间.

由假设 $ g^{\prime \prime}\left(x_{*}\right) \geqslant 0 $. 则有 $ g(x) \geqslant g\left(x_{0}\right)+g^{\prime}\left(x_{0}\right)\left(x-x_{0}\right) $.令 $ x_{0}=\lambda x_{1}+(1-\lambda) x_{2} $.

当 $ x=x_{1} $ 时有 $ g\left(x_{1}\right) \geqslant g\left(x_{0}\right)+g^{\prime}\left(x_{0}\right)\left(x_{1}-x_{0}\right) $,
而 $ x_{1}-x_{0}=x_{1}-\left[\lambda x_{1}+(1-\lambda) x_{2}\right] =\left(x_{1}-x_{2}\right)(1-\lambda)$,因而有
\begin{equation*}
    g\left(x_{1}\right) \geqslant g\left(x_{0}\right)+g^{\prime}\left(x_{0}\right)(1-\lambda)\left(x_{1}-x_{2}\right) \tag{1}
\end{equation*}

当 $ x=x_{2} $ 时有 $ q\left(x_{2}\right) \geqslant g\left(x_{0}\right)+g^{\prime}\left(x_{0}\right)\left(x_{2}-x_{0}\right) $
而 $ x_{2}-x_{0}=x_{2}-\left[\lambda x_{1}+(1-\lambda) x_{2}\right] =\lambda\left(x_{2}-x_{1}\right)$,因而有
\begin{equation*}
    g\left(x_{2}\right) \geqslant g\left(x_{0}\right)+g^{\prime}\left(x_{0}\right) \lambda\left(x_{2}-x_{1}\right) \tag{2}
\end{equation*}

$$
\begin{aligned}
(1)\times \lambda+(2) \times(1-\lambda) &=\lambda g\left(x_{1}\right)+(1-\lambda) g\left(x_{2}\right) \\& \geqslant \lambda g\left(x_{0}\right)+\lambda g^{\prime}\left(x_{0}\right)(1-\lambda)\left(x_{1}-x_{2}\right) \\
& +(1-\lambda) g\left(x_{0}\right)+\lambda g^{\prime}\left(x_{0}\right)(1-\lambda)\left(x_{2}-x_{1}\right) \\
& =g\left(x_{0}\right)
\end{aligned}
$$
即证下凸性,同理可证严格下凸性.
\end{proof}
易知 $ x \geqslant 0 $ 时, $ x^{2},|x|, e^{x}, x \log x $ 均是严格下凸的 .$ \sqrt{x} $ ,$ \log x $ 是严格上凸的.
如 $ g(x)=x \log x $,有 $ g^{\prime}(x)=1+\log x \quad ,g^{\prime \prime}(x)=\frac{1}{x}>0 $

\subsection{熵函数的凸性}
若$\mathscr{X}$是一个固定的集合,记
$$
 \boldsymbol{\tilde{p}}=\left\{\bar{p}=p(x) \mid x \in \mathscr{X}, p(x) \geqslant 0, \sum_{x \in \mathscr{X}} p(x)=1\right\}
$$
是$\mathscr{X}$上的全体概率分布,那么$H(\bar{p}), H(p \| q)$都是$ \boldsymbol{\tilde{p}}$上的函数.
\begin{theorem}
     (1) 熵函数 $ H(\bar{p}) $ 是 $ \boldsymbol{\tilde{p}} $ 上的上凸函数, 即对 $ \forall 0 \leqslant \lambda \leqslant 1 $, $ \forall \overline{p_{1}}, \overline{p_{2}} \in \boldsymbol{\tilde{p}} $, 总有 $ H\left(\lambda \overline{p_{1}}+(1-\lambda) \overline{p_{2}}\right) \geqslant \lambda H\left(\overline{p_{1}}\right)+(1-\lambda) H\left(\overline{p_{2}}\right) $等号成立的充要条件为 $ \lambda=0 $ 或 1 , 或 $ \overline{p_{1}}=\overline{p_{2}} $;
     
(2) 在 $ q $ 固定时,互熵函数 $ H(p \| q) $ 是 $ p $ 的下凸函数,在 $ p $ 固定时, 互熵函数 $ H(p \| q) $ 是 $ q $ 的下凸函数.
\end{theorem}
