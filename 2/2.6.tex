
\section{连续型随机变量的信息量}
\subsection{连续型随机变量的Shannon熵}

\textbf{1. 连续型随机变量}
\begin{definition}
    设 $ \xi $ 是一个随机变量, 在 $ \mathbb{R}=(-\infty,+\infty) $ (或 $ \mathbb{R} $ 的某个区域 $ \mathscr{X} $ ) 中取值, 称 $ \xi $ 是一个连续型随机变量, 它的概率分布函数定义为
$$
F(x)=P_r\{\xi \leqslant x\}, x \in \mathbb{R}
$$
如果 $ F(x) $ 是连续函数, 那么称随机变量 $ \xi $ 具有连续分布,若 $ F(x) $ 的导数存在, 则 $ f(x)=F^{\prime}(x)=\frac{d F(x)}{d(x)} $ 是 $ \xi $ 的概率分布密度函数
\end{definition}

\begin{remark}
  对于概率分布密度函数 $ f(x) $, 有 $ f(x) \geqslant 0 $, 对 $ \forall x \in \mathscr{X} $,
$$
\int_{-\infty}^{+\infty} f(x) d x=1 \text { . }
$$
\end{remark}

\textbf{2. 连续型随机变量的Shannon 熵}
\begin{definition}
    一个概率密度函数为 $ f(x) $ 的连续型随机变量 $ \xi $ 的熵 $ H(\xi) $ 定义为
$$
H(\xi)=-\int_{\mathscr{X}} f(x) \log f(x) d x .
$$
规定 $ 0 \log 0=0 $, 也称为微分熵
\end{definition}
\begin{example}
     (一致分布) 随机变量 $ \xi $ 在 0 到 $ a $ 之间为一致分布, 于是在 0 到 $ a $ 之间的密度为 $ \frac{1}{a} $, 而在其它处的密度均为 0 , 它的熵为
$$
H(\xi)=-\int_{0}^{a} \frac{1}{a} \log \frac{1}{a} d x=\log a
$$
\end{example}
\begin{remark}
    当 $ a<1 $ 时, $ \log a<0 $, 此时熵为负值, 此时并不代表信息的不确定性的大小, 而在物理学中有其意义.
\end{remark}
\begin{example}
  (指数分布) 设 $ \xi \sim p_{\lambda}(x)=\lambda e^{-\lambda x}, x, \lambda \geqslant 0 $, 它的熵为
$$
\begin{aligned}
H(\xi)  =-\int_{0}^{\infty} p_{\lambda}(x) \log p_{\lambda}(x) d x  &=-\int_{0}^{\infty} \lambda e^{-\lambda x} \log \left(\lambda e^{-\lambda x}\right) d x \\
& =-\lambda \int_{0}^{\infty} e^{-\lambda x}(\log \lambda-\lambda x) d x \\
& =-\lambda \int_{0}^{\infty} e^{-\lambda x}(-\lambda x) d x-\int_{0}^{\infty} \lambda e^{-\lambda x} \log \lambda d x \\
& =\int_{0}^{\infty} e^{-\lambda x}(-\lambda x) d(-\lambda x)-\log \lambda \\
& =1-\log \lambda
\end{aligned}
$$
\end{example}
\begin{example}
  (正态分布) 如果 $ \xi \sim N\left(\mu, \sigma^{2}\right) $ ,其中 $ N\left(\mu, \sigma^{2}\right) $ 表示期望为 $ \mu $,均方为 $ \sigma^{2} $ 的正态分布, 那么它的分布密度为
$$
\phi_{\mu, \sigma^{2}}(x)=\frac{1}{\sqrt{2 \pi \sigma^{2}}} \exp \left(-\frac{(x-\mu)^{2}}{2 \sigma^{2}}\right) .
$$
则它的熵为
$$
\begin{aligned}
H\left(\phi_{\mu, \sigma^{2}}\right)&=-\int \phi_{\mu, \sigma^{2}}(x) \log \phi_{\mu, \sigma^{2}}(x) d x \\
&=-\int \phi_{\mu, \sigma^{2}}(x)\left(-\frac{(x-\mu)^{2}}{2 \sigma^{2}}-\log \sqrt{2 \pi \sigma^{2}}\right) d x \\
&=\frac{E\left\{(\xi-\mu)^{2}\right\}}{2 \sigma^{2}}+\frac{1}{2} \log 2 \pi \sigma^{2} \\
&=\frac{1}{2}+\frac{1}{2} \log 2 \pi \sigma^{2} \\
&=\frac{1}{2} \log 2 \pi e \sigma^{2}
\end{aligned}
$$
\end{example}

\subsection{连续型随机变量的联合熵与条件熵}
$ (\xi, \eta) $ 是一对连续型随机变量, 联合分布为 $ p(x, y) $, 边际分布为 $ p(x), p(y) $, 则有如下定义:

(1) 联合熵 $$ H(\xi, \eta)=-\int p(x, y) \log p(x, y) d x d y $$

(2)条件熵 $$ H(\eta \mid \xi)=-\int p(x, y) \log \frac{p(x, y)}{p(x)} d x d y $$

(3)互信息 $$ I(\xi ; \eta)=\int p(x, y) \log \frac{p(x, y)}{p(x) p(y)} d x d y $$









