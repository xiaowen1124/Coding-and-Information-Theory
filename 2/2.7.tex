\section{最大熵原理}
连续型随机变量 $ \xi$, 它的分布密度为$f(x)$, $ H(\xi)=-\displaystyle\int_{\mathscr{X}} f(x) \log f(x) d x $ , $ \xi $ 服从何种概率分布密度函数时, $ H(\xi) $ 最大?

即取合适的概率分布密度函数,使 $ H(\xi) $ 最大.

\subsection{有限区间情形的最大熵(均匀分布时取最大摘)}
设 $ \xi $ 是有限区间 $ \mathscr{X}=(a, b) $ 上取值的随机变量, 约束条件为
$$
\int_{a}^{b} f(x) d x=1
$$
则 $ H(\xi) $ 的最大值为 $ \log (b-a) $, 即 $ H(\xi) \leqslant \log (b-a) $ .

\subsection{半开区间情形的最大熵(指数分布时取最大熵)}

设 $ \xi $ 是在半区间 $ \mathscr{X}=(0, \infty) $ 上取值的随机变量, 约束条件为
$$
\int_{0}^{\infty} f(x) d x=1
$$
其期望固定值为 $ \displaystyle\int_{0}^{\infty} x f(x) d x=\mu>0 $,则 $ H(\xi) $ 的最大值为 $ 1+\log \mu $, 即 $ H(\xi) \leqslant 1+\log \mu,\left(\mu=\frac{1}{\lambda}\right) $ .


\subsection{全区间情形的最大熵(正态分布时取最大熵)}
设 $ \xi $ 是在全区间 $ (-\infty,+\infty) $ 上取值的随机变量,约束条件为
$$
\int_{-\infty}^{+\infty} f(x) d x=1
$$
其期望和方差分别固定为 $ E(\xi)=\mu $ 和 $ D(\xi)=\sigma^{2}>0 $,即 $ \displaystyle\int_{-\infty}^{+\infty} x f(x) d x=\mu $ (数学期望定义),
$$
\int_{-\infty}^{+\infty}(x-\mu)^{2} f(x) d x=\sigma^{2} \text { (方差定义), }
$$
则 $ H(\xi) $ 的最大值为 $ \frac{1}{2} \log \left(2 e \pi \sigma^{2}\right) $,
$$
\text { 即 } H(\xi) \leqslant \frac{1}{2} \log \left(2 e \pi \sigma^{2}\right)
$$