\chapter{Hamming码}

\section{Hamming码的定义}

\subsection{ Hamming码的构造}

通过构造一个 $ r \times n $ 阶校验矩阵来构造 $ q $ 元 Hamming码,令 $ Y_{1}=V(r, q) $, 任取非零向量 $ y_{1} \in Y_{1} $, 令
$$
Y_{2}=Y_{1}-\left\{\alpha y_{1} \mid \alpha \in F_{q}, \alpha \neq 0\right\}
$$
任取非零向量 $ y_{2} \in Y_{2} $, 令 
$$
 Y_{3}=Y_{2}-\left\{\beta y_{2}, \alpha y_{1} \mid \beta, \alpha \in F_{q}, \beta, \alpha \neq 0\right\}
$$
依次下去, 任取 $ y_{i} \in V(r, q) $, 但 $ y_{i} $ 不是前面已取的 $ i-1 $ 个向量中的任何一个的非零倍数, 此过程一直进行到 $ V(r, q) $ 中没有具有此性质的向量为止.

令 $ H=\left(y_{1}^{T}, y_{2}^{T}, \cdots, y_{i}^{T}, \cdots\right) $, 称 $ H $ 为 $ r $ 阶Hamming矩阵, 以 $ H $ 为校验矩阵的线性码称为 $ r $ 阶 $ q $ 元 $ \operatorname{Hamming} $ 码, 记为 $ \operatorname{Ham}(r, q) $.

因为
$$
\left|\left\{\alpha y_{i} \mid \alpha \in F_{q}, \alpha \neq 0\right\}\right|=q-1,
$$
所以校验矩阵 $ H $ 总共有 $ \left(q^{r}-1\right) /(q-1) $ 列, 这也就是说, $ q $ 元 $ \operatorname{Hamming} $ 码 $ \operatorname{Ham}(r, q) $的码长为 $ n=\left(q^{r}-1\right) /(q-1) $.事实上, 上述校验矩阵的构造过程相当于把 $ V(r, q) $ 中的 $ q^{r}-1 $ 个非零向量分成 $ \left(q^{r}-1\right) /(q-1) $ 个类, 使得 $ V(r, q) $ 中任意两个非零向量线性相关的充分必要条件是它们在同一类中. 在每类中任取一个向量作为校验矩阵的列.

应当指出, 对于给定的参数 $ r $ 和 $ q $, 可以选取不同的校验矩阵来定义 $ q $ 元 Ham$ \operatorname{ming} $ 码 $ \operatorname{Ham}(r, q) $, 但通过列的置换以及某些列与 $ F_{q} $ 中非零元素的相乘, 任意两个不同的校验矩阵可以相互转化. 因此, 在等价的意义下, $ q $ 元 $ \operatorname{Hamming} $ 码 $ \operatorname{Ham}(r, q) $是唯一的.

\section{Hamming码的性质}
\begin{theorem}
 $ q $ 元Hamming 码 $ \operatorname{Ham}(r, q) $ 是 $ q $ 元 $ [n, n-r, 3] $ 线性码, 其中 $ n=\frac{q^{r}-1}{q-1} $.
\end{theorem}
\begin{proof}
$ q $ 元 Hamming码 $ \operatorname{Ham}(r, q) $ 的检验阵 $ H_{r \times n} $ 为 $ \operatorname{Ham}(r, q)^{\perp} $ 的生成阵, 故 $ \operatorname{Ham}(r, q) $ 为 $ [n, n-r] $ 线性码, 由 $ H $ 的构造方法知 $ V(r, q) $ 中共有 $ q^{r}-1 $ 个非零向量.$\forall x \in V(r, q) \quad x \neq 0, \quad \alpha \in F_{q}, \quad \alpha \neq 0$,
$ \alpha x$ 与 $ x $ 线性相关, $ \alpha x $ 共有 $ q-1 $ 个, 于是可将 $ V(r, q) $ 中 $ q^{r}-1 $ 个非零向量分组,每组有 $ q-1 $ 个,故共有 $ \frac{q^{r}-1}{q-1} $ 组, 每组取一个向量作为 $ H $ 的 1 列,共有 $ \frac{q^{r}-1}{q-1} $ 列, 故 $ n=\frac{q^{r}-1}{q-1} $, 每个向量 $ x $ 为 $ r $ 长, 故 $ x^{T} $ 作为列, $ H $ 是 $ r $ 行 $ n $ 列矩阵.
由 $ H $ 的构造可知, $ H $ 的任意两列线性无关.故令 $ H=\left(h_{1}, h_{2}, \cdots, h_{n}\right), \quad $ 有 $ h_{1}+h_{2} \neq 0 $ (否则相关)故 $ \exists i $, 有 $ h_{1}+h_{2}=\alpha h_{i} $ (因非零向量都在 $ H $ 的列分组中)因此 $ H $ 中存在 3 列线性相关, 从而 $ \operatorname{Ham}(r, q) $ 的最小距离为 3 .
\end{proof}

\begin{example}
  $ V(2,3) $ 中, $ r=2, q=3,|V(2,3)|=3^{2}=9 $, 故 $ V(2,3) $中有8个非零向量,按Hamming码的构造方法进行分组, 
$$
\begin{array}{l}
\text {  取 }\left(\begin{array}{l}
0 \\
1
\end{array}\right) \text {, 则 } \alpha\left(\begin{array}{l}
0 \\
1
\end{array}\right)=\left(\begin{array}{l}
0 \\
0
\end{array}\right),\left(\begin{array}{l}
0 \\
1
\end{array}\right),\left(\begin{array}{l}
0 \\
2
\end{array}\right) \cdot\left\{\left(\begin{array}{l}
0 \\
1
\end{array}\right),\left(\begin{array}{l}
0 \\
2
\end{array}\right)\right\} \text {. } \\
\text { 取 }\left(\begin{array}{l}
1 \\
0
\end{array}\right) \text {, 则 }\left\{\left(\begin{array}{l}
1 \\
0
\end{array}\right),\left(\begin{array}{l}
2 \\
0
\end{array}\right)\right\} \\
\text { 取 }\left(\begin{array}{l}
1 \\
1
\end{array}\right) \text {, 则 }\left\{\left(\begin{array}{l}
1 \\
1
\end{array}\right),\left(\begin{array}{l}
2 \\
2
\end{array}\right)\right\} \\
\text { 取 }\left(\begin{array}{l}
1 \\
2
\end{array}\right) \text {, 则 }\left\{\left(\begin{array}{l}
1 \\
2
\end{array}\right),\left(\begin{array}{l}
2 \\
1
\end{array}\right)\right\} \\
\end{array}
$$
于是 $ \operatorname{Ham}(2,3) $ 的校验矩阵为 $ \left(\begin{array}{llll}0 & 1 & 1 & 1 \\ 1 & 0 & 1 & 2\end{array}\right) $ .

$ n=\frac{q^{r}-1}{q-1}=\frac{3^{2}-1}{3-1}=\frac{8}{2}=4 $, 码长为 4 .
$ \operatorname{Ham}(r, q) $ 是一个 $ [4,2,3] $ 码.
\end{example}


\begin{example}
 $ V(3,3) $ 中非零码字个数为 $ 3^{3}-1=26 $ 个, $ H $ 的列数为$\frac{q^{r}-1}{q-1}=\frac{26}{2}=13 .$
$$
\begin{array}{l}
\left\{\left(\begin{array}{l}
0 \\
0 \\
1
\end{array}\right),\left(\begin{array}{l}
0 \\
0 \\
2
\end{array}\right)\right\},\left\{\left(\begin{array}{l}
0 \\
1 \\
0
\end{array}\right),\left(\begin{array}{l}
0 \\
2 \\
0
\end{array}\right)\right\},\left\{\left(\begin{array}{l}
1 \\
0 \\
0
\end{array}\right),\left(\begin{array}{l}
2 \\
0 \\
0
\end{array}\right)\right\}, \\
\left\{\left(\begin{array}{l}
1 \\
1 \\
0
\end{array}\right),\left(\begin{array}{l}
2 \\
2 \\
0
\end{array}\right)\right\},\left\{\left(\begin{array}{l}
0 \\
1 \\
1
\end{array}\right),\left(\begin{array}{l}
0 \\
2 \\
2
\end{array}\right)\right\},\left\{\left(\begin{array}{l}
1 \\
0 \\
1
\end{array}\right),\left(\begin{array}{l}
2 \\
0 \\
2
\end{array}\right)\right\},\left\{\left(\begin{array}{l}
1 \\
1 \\
1
\end{array}\right),\left(\begin{array}{l}
2 \\
2 \\
2
\end{array}\right)\right\}, \\
\left\{\left(\begin{array}{l}
0 \\
1 \\
2
\end{array}\right),\left(\begin{array}{l}
0 \\
2 \\
1
\end{array}\right)\right\},\left\{\left(\begin{array}{l}
1 \\
0 \\
2
\end{array}\right),\left(\begin{array}{l}
2 \\
0 \\
1
\end{array}\right)\right\},\left\{\left(\begin{array}{l}
1 \\
2 \\
0
\end{array}\right),\left(\begin{array}{l}
2 \\
1 \\
0
\end{array}\right)\right\}, \\
\left\{\left(\begin{array}{l}
1 \\
1 \\
2
\end{array}\right),\left(\begin{array}{l}
2 \\
2 \\
1
\end{array}\right)\right\},\left\{\left(\begin{array}{l}
1 \\
2 \\
1
\end{array}\right),\left(\begin{array}{l}
2 \\
1 \\
2
\end{array}\right)\right\},\left\{\left(\begin{array}{l}
1 \\
2 \\
2
\end{array}\right),\left(\begin{array}{l}
2 \\
1 \\
1
\end{array}\right)\right\} .
\end{array}
$$
于是有 $ H=\left(\begin{array}{lllllllllllll}0 & 0 & 0 & 0 & 1 & 1 & 1 & 1 & 1 & 1 & 1 & 1 & 1 \\ 0 & 1 & 1 & 1 & 0 & 0 & 0 & 1 & 1 & 1 & 2 & 2 & 2 \\ 1 & 0 & 1 & 2 & 0 & 1 & 2 & 0 & 1 & 2 & 0 & 1 & 2\end{array}\right) $.  $ \operatorname{Ham}(3,3) $ 是一个 $ [13,10,3] $ 码.
\end{example}


\begin{theorem} 
$ q $ 元 $ \operatorname{Hamming} $ 码 $ \operatorname{Ham}(r, q) $ 是完备的(达到Hamming界).
\end{theorem}
\begin{proof}
$ \operatorname{Ham}(r, q) $ 的最小距离为 3 , 故可纠正 1 个错误,我们有
$$
\begin{array}{l}
M\left(\left(\begin{array}{l}
n \\
0
\end{array}\right)+\left(\begin{array}{l}
n \\
1
\end{array}\right)(q-1)+\left(\begin{array}{l}
n \\
2
\end{array}\right)(q-1)^{2}+\cdots+\left(\begin{array}{c}
n \\
t
\end{array}\right)(q-1)^{t}\right) \\
=q^{n-r}\left(1+\left(\begin{array}{l}
n \\
1
\end{array}\right)(q-1)\right) \\
=q^{n-r}\left[1+\frac{q^{r}-1}{q-1}(q-1)\right] \\
=q^{n-r} \cdot q^{r}=q^{n}
\end{array}
$$
于是Hamming码是完备码.
\end{proof}


