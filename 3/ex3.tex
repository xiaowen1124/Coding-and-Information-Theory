\section{习题课}
\subsection{基本概念}

1. 信源编码: $ f: \mathscr{X}\left(\right. $ 或 $ \left.\mathscr{X}^{n}\right) \rightarrow \mathscr{U}^{*} $ 的映射称为一个信源编码.

2. 1-1码: $ f $ 是一个信源编码. 若对 $ \forall x, x^{\prime} \in \mathscr{X}, x \neq x^{\prime} $ 都有 $ f(x) \neq f\left(x^{\prime}\right) $,称为 $ f $ 为一个1-1码,或称 $ f $ 具有 1-1性.

3. 扩张编码 (变长和定长)

4. 唯一可译码: 若 $ f $ 的扩张编码 $ f^{*} $ 是 1-1映射,则称 $ f $ 是唯一可译码.

5. 即时码: 如果对于任意一个码字串, 从左到右. 如果码字 $ c_{i} $ 出现就译为其相应的消息字母, 就称这种码为即时码, 否则称为非即时码.

6. 前缀码:设 $ a^{(k)}=\left(a_{1}, a_{2}, \cdots, a_{k}\right), \quad b^{\left(k^{\prime}\right)}=\left(b_{1}, b_{2}, \cdots, b_{k^{\prime}}\right) $若 $ k \leq k^{\prime} $ 且有 $ \left(a_{1}, \cdots, a_{k}\right)=\left(b_{1}, \cdots, b_{k}\right) $, 则称 $ a^{(k)} $ 为 $ \left.b^{(k '}\right) $ 的前缀. 若在一个码中, 任何一个码字都不是另一个码字的前缀, 即任意码字 $ c_{i} $, 都不是 $ c_{j} $ 的前缀, 称这种码为前缀码.

7. 平均码长: $ \mathscr{S}=(\mathscr{X}, p(x)), \bar{p}=\left\{p_{1}, p_{2}, \cdots, p_{a}\right\} $, $ \ell_{f}(x) $ 为码字 $ f(x) $ 的长度, 称 $ L(\mathscr{S}, f)=\sum\limits_{i=1}^{a} p(x) \ell_{f}(x) $ 为 $ f $ 的平均码长.


\subsection{编码问题与基本结论}

\textbf{1. 变长码的编码问题}

信源固定,构造变长即时码 $ f_{0} $, 使得对任意变长即时码 $ f $ 有
$$
L\left(\mathscr{S}, f_{0}\right) \leq L(\mathscr{S}, f)
$$
此时,称 $ f_{0} $ 为最优变长即时码.

最优码构造方法: Huffman编码算法.

\textbf{2. 定长编码(信源序列)的编码问题}

$ \mathscr{S}^{n}=\left(\mathscr{X}^{n}, p^{(n)}\left(x^{(n)}\right)\right), \quad $ 求最小可达速率.

信源定长码的编码定理:设 $ \mathscr{S}^{n} $ 是一个由 $ \mathscr{S} $ 决定的无记忆信源, 那么它的最小可达速率为
$$
R_{0}=H(\xi)=H\left(p_{1}, p_{2}, \cdots, p_{a}\right),
$$
其中 $ H(\xi) $ 是 $ \xi $ 的熵.

\textbf{3.Kraft不等式(即时码存在的必要条件)}
$$
\sum_{k=1}^{a} \frac{1}{r^{\ell_{k}}} \leq 1 \quad C=\left\{c_{1}, c_{2}, \cdots, c_{a}\right\}
$$
$ \ell_{1}, \ell_{2}, \cdots, \ell_{a} $ 为码字的长度, $ r=|\mathscr{U}| $.
反之, 若 $ \left\{\ell_{1}, \ell_{2}, \cdots, \ell_{a}\right\} $ 满足上式, 则一定存在以 $ \ell_{1}, \ell_{2}, \cdots, \ell_{a} $ 为码字长度的即时码(构造方法).

\textbf{4.满足Kraft不等式的码不一定是即时码,反例需知道.}

\textbf{5. 最优变长码平均码长的上、下界}

定理:对于已给信源 $ \mathscr{S}=\{\mathscr{X}, p(x)\} $, 它的 $ r $ 元最优变长即时码 $ f_{0} $ 有
$$
H_{r}\left(p_{1}, \cdots, p_{a}\right) \leq L\left(\mathscr{S}, f_{0}\right)<H_{r}\left(p_{1}, \cdots, p_{a}\right)+1
$$
成立.

\subsection{课后习题}
\begin{exercise}
下面的码是否是即时码? 是否是唯一可译码?\\
(1) $ C=\{0,10,1100,1101,1110,1111\} $.\\
(2) $ C=\{0,10,110,1110,1011,1101\} $.
\end{exercise}
\begin{solution}
 (1) C 是前缀码, 故是即时码, 从而是唯一可译码.
 
(2) C不是前缀码, 因为码字10是码字1011的前缀, 故C不是即时码.C不是唯一可译码.
字符串
$$
\frac{0}{a} \frac{10}{b} \frac{110}{c} \frac{1110}{d} \frac{1011}{e} \frac{1101}{\mathrm{f}} $$
$$
\frac{0}{a} \frac{1011}{e} \frac{0}{a} \frac{1110}{d} \frac{1011}{e} \frac{1101}{\mathrm{f}}
$$
同一字符串的还原消息为两个, 不是唯一可译码.
\end{solution}


\begin{exercise}
 判断是否存在即时码具有以下的基数和码字长度, 如果有, 试构造出一个这样的码.\\
(1) $ r=2 $, 长度: 1,3,3,3,4,4.\\
(2) $ r=3 $ ,长度: 1, 1,2,2,3,3,3.\\
(3) $ r=5 $, 长度: $ 1,1,1,1,1,8,9 $.

\end{exercise}
\begin{solution}
(1) $ \frac{1}{2}+3 \times \frac{1}{2^{3}}+2 \times \frac{1}{2^{4}}=1 $,故满足Kraft不等式. 即时码存在.
$$
\begin{array}{lllll}
u_{1,1}=0 & 0 & & &\\
u_{3,1,1}=1 & u_{3,1,2}=0 & u_{3,1,3}=0 & (1,0,0) &\\
u_{3,2,1}=1 & u_{3,2,2}=0 & u_{3,2,3}=1 & (1,0,1) &\\
u_{3,3,1}=1 & u_{3,3,2}=1 & u_{3,3,3}=0 & (1,1,0) &\\
u_{4,1,1}=1 & u_{4,1,2}=1 & u_{4,1,3}=1 & u_{4,1,4}=0 & (1,1,1,0)\\
u_{4,2,1}=1 & u_{4,2,2}=1 & u_{4,2,3}=1 & u_{4,2,4}=1& (1,1,1,1)
\end{array}
$$
故此即时码为
$$
\{0,100,101,110,1110,1111)\}
$$

(2)$ 2 \times \frac{1}{3}+2 \times \frac{1}{3^{2}}+3 \times \frac{1}{3^{3}}=1 $ 故满足 Kraft 不等式, 即时码存在.
$$
u_{1,1}=0 \quad u_{1,2}=1 \quad 0,1
$$
$$
\begin{array}{llrl}
u_{2,1,1}=2 & u_{2,1,2}=0 & (2,0) & \\
u_{2,2,1}=2 & u_{2,2,2}=1 & (2,1) & \\
u_{3,1,1}=2 & u_{3,1,2}=2 & u_{3,1,3}=0 & (2,2,0) \\
u_{3,2,1}=2 & u_{3,2,2}=2 & u_{3,2,3}=1 & (2,2,1) \\
u_{3,3,1}=2 & u_{3,3,2}=2 & u_{3,3,3}=2 & (2,2,2)
\end{array}
$$

故此即时码为
$$
\{0,1,20,21,220,221,222\}
$$

(3)$ 5 \times \frac{1}{5}+\frac{1}{5^{8}}+\frac{1}{5^{9}}>1 $, 故这样的即时码不存在.
\end{solution}


\begin{exercise}
在证明 Kraft不等式中,我们说
$$
\frac{\alpha_{1}}{r}+\frac{\alpha_{2}}{r^{2}}+\cdots+\frac{\alpha_{n}}{r^{n}} \leq 1
$$
等价于Kraft不等式.证明这个结果.
\end{exercise}
\begin{solution}
令 $ \alpha_{j} $ 表示 $ \ell_{i}=j $ 的 $ i $ 的个数, 即长度为 $ j $ 的码字个数为 $ \alpha_{j} $ 个, 故长度为 1 的个数为 $ \alpha_{1} $ 个, 长度为 2 的个数为 $ \alpha_{2} $ 个, $ \cdots $, 长度为 $ n $ 的个数为 $ \alpha_{n} $ 个.
$$
\begin{aligned}
\sum_{i=1}^{n} \frac{1}{r^{\ell_{i}}} & =\alpha_{1} \cdot \frac{1}{r^{1}}+\alpha_{2} \cdot \frac{1}{r^{2}}+\cdots+\alpha_{j} \cdot \frac{1}{r^{j}}+\cdots+\alpha_{n} \cdot \frac{1}{r^{n}} \\
& =\sum_{i=1}^{n} \frac{\alpha_{i}}{r^{i}} \leq 1
\end{aligned}
$$
\end{solution}


\begin{exercise}
令 $ C $ 是一个即时码, 试证明下列命题等价.\\
(1) $ C $ 是最大即时码, 即没有码字能够添入 $ C $ 中而令 $ C $ 仍保持即时性.\\
(2) 任意码元素的有限串都是某个码字串的前缀.\\
(3)Kraft不等式中的等号成立.
\end{exercise}
\begin{solution}
只证明 $ (1) \Longleftrightarrow(3) $.

$ \Leftarrow $ : 先证明若 C不是最大即时码, 则Kraft不等式中等号不成立.若 $ C $ 不是最大即时码, 则在码 $ C $ 中可至少添入一个码字, 成为一个新的即时码 $ C_{1} $, 假设 $ C=\left\{c_{1}, c_{2}, \cdots, c_{a}\right\}, c_{i} $ 的码长为 $ \ell_{i} $, $ i=1, \cdots, a $, 设添入的码字为 $ C_{a+1} $, 长度为 $ \ell_{a+1} $, 因 $ C_{1} $ 仍为即时码, 故有
$$
\sum_{i=1}^{a+1} \frac{1}{r^{\ell_{i}}}=\sum_{i=1}^{a} \frac{1}{r^{\ell_{i}}}+\frac{1}{r^{\ell_{a}+1}} \leq 1
$$
从而有 $ \sum\limits_{i=1}^{a} \frac{1}{r^{\ell_{i}}}<1 $. 故 Kraft不等式中等号不成立.

$ \Rightarrow $ : 若Kraft不等式中等号不成立, 令 $ C=\left\{c_{1}, c_{2}, \cdots, c_{a}\right\} $, 对应码字长为 $ \left\{\ell_{1}, \ell_{2}, \cdots, \ell_{a}\right\} $. 此时有$\sum\limits_{k=1}^{a} \frac{1}{r^{\ell_{k}}}<1$
$.
\text { 令 } \ell=\operatorname{lnt}\left(\log _{r}\left(1-\sum\limits_{k=1}^{a} \frac{1}{r^{\ell_{k}}}\right)\right)+1$,
则有 $ \sum\limits_{k=1}^{a} \frac{1}{r^{\ell_{k}}}+\frac{1}{r^{\ell}} \leq 1 $ 故 $ \left\{\ell_{1}, \ell_{2}, \cdots, \ell_{a}, \ell\right\} $ 满足 $ \operatorname{Kraft} $ 不等式于是可构造即时码 $ C^{\prime}=\left\{c_{1}, c_{2}, \cdots, c_{a}, c^{\prime}\right\} $, 与 $ C $ 是最大即时码矛盾.
\end{solution}


\begin{exercise}
 对下面给定的概率分布和基数, 找出一个Huffman编码, 并求平均码长.
$$
p=\{0.3,0.1,0.1,0.1,0.1,0.06,0.05,0.05,0.05,0.04,0.03,0.02\},
\quad r=2 .
$$
\end{exercise}
\begin{solution}
(1) 先确定 ${k} $ 的值.
$$
k=\operatorname{I n t}_{+}\left(\frac{a-1}{r-1}\right)=\frac{11}{1}=11 .
$$
(2)再确定第1列最后几个分量相加.
$$
\begin{array}{l}
a-(k-1) r+k-1 \\
=12-(11-1) \times 2+11-1=2 .
\end{array}
$$
于是我们构造Huffman编码为
\begin{center}
\begin{tabular}{ll||ll||ll||ll||ll||ll} 
\hline
概率 & 码 & 概率 & 码 & 概率 & 码 &概率 & 码 & 概率 & 码 & 概率 & 码 \\
\hline
0.3 & 00 & 0.3 & 00 & 0.3 & 00 &0.3 & 00 & 0.3 & 00 & 0.3 & 00 \\
0.1 & 111 & 0.1 & 111 & 0.1 & 111 &$ \boxed{0.1} $ & 110 & $ \boxed{0.11} $ & 011 & $ \boxed{0.19} $ & 010 \\
0.1 & 100 & 0.1 & 100 & 0.1 & 100 &0.1 & 111 & 0.1 & 110 & 0.11 & 011 \\
0.1 & 101 & 0.1 & 101 & 0.1 & 101 &0.1 & 100 & 0.1 & 111 & 0.1 & 110 \\
0.1 & 0100 & 0.1 & 0100 & 0.1 & 0100 &0.1 & 101 & 0.1 & 100 & 0.1 & 111 \\
0.06 & 0110 & 0.06 & 0110 & $ \boxed{0.09} $ & 0101 &0.1 & 0100 & 0.1 & 101 & 0.1 & 100 \\
0.05 & 1100 & $ \boxed{0.05} $ & 0111 & 0.06 & 0110 &0.09 & 0101 & 0.1 & 0100 & 0.1 & 101 \\
0.05 & 1101 & 0.05 & 1100 & 0.05 & 0111 &0.06 & 0110 & 0.09 & 0101 & & \\
0.05 & 01010 & 0.05 & 1101 & 0,05 & 1100 &0.05 & 0111 & & & &\\
0.04 & 01011 & 0.05 & 01010 & 0.05 & 1101 & & & & & &\\
0.03 & 01110 & 0.04 & 01011 & & & & & & & &\\
0.02 & 01111 & & & & & & & & &\\
\hline
\hline
概率 & 码 & 概率 & 码 & 概率 & 码 &概率 & 码 & 概率 & 码 & & \\
\hline
0.3 & 00 & 0.3 & 00 & 0.3 & 00 &0.4 & 1 & $ \boxed{0.6} $ & 0 & &\\
0.2 & 10 & 0.2 & 10 & $ \boxed{0.3} $ & 01 &0.3 & 00 & 0.4 & 1 & & \\
0.19 & 010$\quad$ & $ \boxed{0.2} $ & 11 & 0.2 & 10 &0.3 & 01 & & & & \\
0.11 & 011 & 0.19 & 010$\quad$ & 0.2 & 11 & & & & & & \\
0.1 & 110 & 0.11 & 011$\quad$ & & & & & & &\\
0.1 & 111 & & & & & & & & & &\\
\hline
\end{tabular}
\end{center}
$$ L(\mathscr{S}, f)=0.14 \times 5+0.26 \times 4+0.3 \times 3+0.3 \times 2=3.24 $$
\end{solution}


\newpage
\begin{mdframed}[frametitle={等价证明}, frametitlerule=true, frametitlebackgroundcolor=yellow!20]

练习题等价命题的证明:

 \textbf{从 (1) 到 (2):}假设 $C$ 是最大即时码,意味着不能向 $C$ 添加更多的码字而保持其即时性.我们需要证明,任意码元素的有限串都是某个码字串的前缀.

由于 $C$ 是最大即时码,如果存在一个码元素的串,它不是任何码字串的前缀,那么我们可以将这个串作为一个新的码字添加到 $C$ 中,而不违反即时性.这与 $C$ 是最大即时码的假设矛盾.因此,任意码元素的有限串必须是某个码字串的前缀.

 \textbf{从 (2) 到 (3):}假设任意码元素的有限串都是某个码字串的前缀.这意味着码字集覆盖了所有可能的码元组合,形成了一种“完整”的编码系统,没有未被利用的“空间”.

Kraft不等式表达了对于长度可变的即时码,其码字长度的集合 $\{l_1, l_2, \ldots, l_n\}$ 必须满足以下条件:
\[
\sum_{i=1}^{n} r^{-l_i} \leq 1
\]
如果任意码元素的有限串都是某个码字串的前缀,这意味着编码系统利用了所有可用的编码“空间”,不留下任何“空隙”.因此,Kraft不等式中的等号必须成立,否则还有“空间”可以添加更多的码字而不违反即时性,这与假设矛盾.

\textbf{ 从 (3) 到 (1):}假设Kraft不等式中的等号成立,即

\[
\sum_{i=1}^{n} r^{-l_i} = 1
\]

这表明编码系统完美地匹配了编码空间的容量,没有未被利用的部分.在这种情况下,不能添加更多的码字而不增加现有码字的长度,因为这将违反Kraft不等式,导致总和超过1.因此,如果Kraft不等式中的等号成立,那么 $C$ 必须是最大即时码,因为没有余地添加更多的码字而保持即时性.
\end{mdframed}








