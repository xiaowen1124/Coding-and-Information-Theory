\section{Huffman信源编码算法}
\subsection{Huffman编码的实例分析}
变长最优码就是平均码字长度取最小值的即时码. 本节我们给出一种有效地构造最优编码方案的算法 —— Huffman 编码.

\begin{example}
    已知信源概率分布为 $ \overline{p}=\{0.24,0.20,0.18,0.13,0.10,0.06,0.05,0.03,0.01\} $, 如取信号字母表 $ \mathscr{U}=\{0,1,2,3\} $,求信源的Huffman编码.
\end{example}

Huffman编码方案的主要运算步骤是构造Huffman数据压缩表与Huffman编码表. 它的运算步骤如下.

1. 先构造 Huffman 数据压缩表\\
(1) 首先把概率分布 $ \overline{p} $ 的概率按大小降序排列, 并作为把它们作为 Huffman数据压缩表第 1 列, 该列长度为 $ a=9 $.\\
(2) 把第一列的 3 个最小的概率相加, 得到新的一列概率分布, 重新按降序排列, 成为Huffman数据压缩表中的第三列. 这时第三列的长度为 7 , 相加后的数为 0.09 , 我们用方框标出.\\
(3) 把第三列的 4 个最小的概率相加, 得到新的一列概率分布, 重新按降序排列, 成为Huffman数据压缩表中的第五列. 这时第五列的长度为 4 , 相加后的数为 0.38 , 我们用方框标出.

2. Huffman 数据压缩表构造 Huffman 编码表\\
(1)因Huffman数据压缩表的第五列的概率分布只有 4 个行, 因此它们的编码正好是 $ 0,1,2,3 $. 把这 4 个数填入Huffman 编码表的第六列. 因此, 第六列是一个码
长为 1 的编码.\\
(2) 在第五列中, 带方框的概率为 0.38 , 它对应的第六列的编码为 0 , 而 0.38 是由第三列的 $ 0.13,0.10,0.09,0.06 $ 这 4 个数相加而成. 这样我们构造第三列概率分布的编码为: 第三列的 $ 0.13,0.10,0.09,0.06 $ 这 4 个数的编码是在 0.38 的编码 0 后边延长 1 个数, 它们分别为 $ 00,01,02,03 $. 第三列中 $ 0.24,0.20,0.18 $ 这 3 个数的编码与第五列的编码相同, 仍为 $ 1,2,3 $. 把 $ 0.24,0.20,0.18,$ $0.13,0.10,0.09,0.06 $ 这 7 个数的编码 $ 1,2,3,00,01,02,03 $ 列入Huffman编码表的第四列.\\
(3) 在表的第三列中, 带方框的概率为 0.09 , 它对应的第四列的编码为 02 , 而 0.09 是由第一列的 $ 0.05,0.03,0.01 $ 这 3 个数相加而成. 这样我们构造第一列概率分布的编码为第一列的前 6 个数的编码与第三列所对应的编码相同. 第一列中 $ 0.05,0.03,0.01 $ 这 3 个数的编码是在第 3 列的 0.09 的编码 02 后边延长 1 个数, 因此它们分别为 $ 020,021,022 $. 把第一列 9 个概率的编码列入Huffman编码表的第二列. 最终完成Huffman编码表. 各项计算结果见表 .

以上过程为 Huffman编码方案. 主要原则是概率大的信源字母对应长度短的码字, 概率小的信源字母对应长度大的码字以保证平均码长尽量小.\textbf{频率高的元素使用较短的编码,频率低的元素使用较长的编码.这样做可以减少整体编码的平均长度,从而达到压缩数据的目的.}


\begin{center}
\begin{tabular}{ll||ll||ll}
\hline 概事 & 码 & 概率 & 码 & 概率 & 码 \\
\hline 0.24 & 1 & 0.24 & 1 & \boxed{0.38} & 0 \\
0.20 & 2 & 0.20 & 2 & 0.24 & 1 \\
0.18 & 3 & 0.18 & 3 & 0.20 & 2 \\
0.13 & 00 & 0.13 & 00 & 0.18 & 3 \\
0.10 & 01 & 0.10 & 01 & & \\
0.06 & 03 & \boxed{0.09} & 02 & & \\
0.05 & 020 & 0.06 & 03 & & \\
0.03 & 021 & & & \\
0.01 & 022 & & & & \\
\hline
\end{tabular}
\end{center}

\subsection{Huffman编码的一般算法}
考虑一般情形. 令码字母表为 $ \mathscr{U}=\{0,1, \cdots, r-1\} $. 信源概率分布为 $ \overline{p}=\left(p_{1}, p_{2}, \cdots, p_{a}\right) $. 构造Huffman 编码的步骤:

1. 若 $ a \leqslant r $, 则取 $ f\left(x_{i}\right)=i-1 $ 即可, $ i=1, \cdots, a $

2. 对于 $ a>r $, 我们观察上述例子的编码过程, 知编码表的最后一列序含 $ r $ 个元素. 设编码表为 $ 2 k $ 列,则有第 $ 2 k $ 列序含 $ r $ 个元,第 $ 2 k-2 $ 列序含 $ 2 r-1 $ 个元, $ 2 k-4 $ 列序含 $ 3 r-2 $ 个元, 即是一个以 $ r-1 $ 为公差的等差数列,故编码表中的第 4 列序含 $ (k-1) r-(k-2) $ 个元.第2列序应含 $ k r-(k-1) $ 个元(确定 $ {k} $ 的值)因此: 

(1) 确定 $ {k} $ 的值. $ k r-k+1 \geq a $, 故 $ k \geq \frac{a-1}{r-1} $, 
${k} $ 取 $ \geq \frac{a-1}{r-1} $ 的最小正整数. 故令 $ k=\operatorname{Int}_{+}\left(\frac{a-1}{r-1}\right) $.

(2)确定第1列中应把最后的几个分量相加,再按大小顺序排列,应该为 $ a-[(k-1) r-(k-2)]+1 $ 个分量相加.

(3)从第3列开始每次将最后 $ r $ 个分量相加,并按大小排序放入下一奇数列, 并将相加所得的概率用框标出. 于是得到了 Huffman压缩表.

(4)按例中编码规则去写出编码表即可.

上述算法称为 Huffman编码算法,得到的编码
$
C=\left\{c_{1}, c_{2}, \cdots, c_{a}\right\}
$
称为 Huffman码.

\begin{example}
    设 $ \mathscr{S}=(\mathscr{X}, p(x)), \bar{p}= $ $ \{0.3,0.1,0.1,0.1,0.1,0.06,0.05,0.05,0.05,0.04,0.03,0.02\} $, 试构造 $ \mathscr{S} $ 上的3元Huffman码并求平均码长.

    解: (1) 先确定 $ {k} $ 的值
$$
k=\operatorname{I n t}_{+}\left(\frac{a-1}{r-1}\right)=\frac{11}{2}=6
$$
(2)再确定第1列最后几个分量相加
$$
\begin{array}{l}
a-(k-1) r+k-1 \\
=12-(6-1) \times 3+6-1 \\
=12-15+6-1=2
\end{array}
$$
于是我们构造Huffman编码为
\begin{center}
\begin{tabular}{ll||ll||ll||ll||ll||ll} 
\hline
概率 & 码 & 概率 & 码 & 概率 & 码 &概率 & 码 & 概率 & 码 & 概率 & 码 \\
\hline
0.3 & 1 & 0.3 & 1 & 0.3 & 1 &0.3 & 1 & 0.3 & 1 &$\boxed{0.4} $ & 0 \\
0.1 & 02 & 0.1 & 02 & $ \boxed{0.14} $ & 01 &$ \boxed{0.16} $ & 00 & $ \boxed{0.3} $ & 2 & 0.3 & 1 \\
0.1 & 20 & 0.1 & 20 & 0.1 & 02 &0.14 & 01 & 0.16 & 00 & 0.3 & 2 \\
0.1 & 21 & 0.1 & 21 & 0.1 & 20 &0.1 & 02 & 0.14 & 01  & \\
0.1 & 22 & 0.1 & 22 & 0.1 & 21 &0.1 & 20 & 0.1 & 02 & \\
0.06 & 000 & 0.06 & 000 & 0.1 & 22 &0.1 & 21 & & & \\
0.05 & 001 & 0.05 & 001 & 0.06 & 000 &0.1 & 22 & & & \\
0.05 & 002 & 0.05 & 002 & 0.05 & 001 & & & &\\
0.05 & 010 & 0.05 & 010 & 0.05 & 002 & & & &\\
0.04 & 012 & $\boxed{0.05}$ & 011 & & & & & &\\
0.03 & 0110 & 0.04 & 012 & & & & & &\\
0.02 & 0111 & & & & & & & &\\
\hline
\end{tabular}
\end{center}

$ L(\mathscr{S}, f)=0.3 \times 1+0.4 \times 2+0.25 \times 3+0.05 \times 4=2.05 $.

\end{example}

\begin{example}
  $ \bar{p}=(0.2,0.1,0.1,0.3,0.1,0.2), r=3 $, 求Huffman编码
    
    解: (1) $ k=\operatorname{lnt}_{+}\left(\frac{a-1}{r-1}\right)=\operatorname{Int}_{+}\left(\frac{6-1}{3-1}\right)=3 $
    
(2)确定第1列后几个分量相加 (最后分量相加的个数): $ a-(k-1) r+k-1=6-(3-1) \times 3+3-1=2 $. 于是有
\begin{center}
\begin{tabular}{ll||ll||ll} 
\hline
概率 & 码 & 概率 & 码 & 概率 & 码 \\
\hline
0.3 & 1 & 0.3 & 1 & $ \boxed{0.5} $ & 0 \\
0.2 & 2 & 0.2 & 2 & 0.3 & 1 \\
0.2 & 00 & 0.2 & 00 & 0.2 & 2 \\
0.1 & 02 & $ \boxed{0.2} $ & 01 & & \\
0.1 & 010 & 0.1 & 02 & & \\
0.1 & 011 & & & &\\
\hline
\end{tabular}
\end{center}

\end{example}











