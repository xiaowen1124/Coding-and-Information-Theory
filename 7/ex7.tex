\section{习题课}

\subsection{本章小结}
1. 基本概念\\
线性码, 生成矩阵, 校验阵, 标准阵(列)译码,伴随式,对偶码, 重量分布多项式, MDS码.

2. 基本结论\\
(1) 校验阵与最小距离的关系\\
(2) 线性码的 $ G-V $ 界. $ d \leq n-k+1 $\\
(3) 标准阵译码方法\\
(4) Mac Williams恒等式(二元情形)

\subsection{课后习题}
\begin{exercise}
    一个二元 $ (11,24,5) $ 码会是线性码吗? 为什么?
\end{exercise}
\begin{solution}
    二元 $ (11,24,5) $ 码不会是线性码. 二元线性码的码字个数应该是 $ 2^{k} $,而 24 不是 2 的幂.
\end{solution}


\begin{exercise}
 设 $ E_{n} $ 是 $ V(n, 2) $ 中所有具有偶数重量的向量的集合. 证明: $ E_{n} $ 是线性码, 确定 $ E_{n} $ 的参数 $ [n, k, d] $ 以及其标准型的生成矩阵.
\end{exercise}
\begin{solution}
    证明: $ \forall x, y \in E_{n}, d(x, y)=\omega(x-y) $,
$ \omega(x-y)=\omega(x)+\omega(y)-2 \omega(x \cap y) $. 故 $ \omega(x+y) $ 为偶数, $ x+y=x-y, x+y \in E_{n}, E_{n} $ 是线性码.
由上一章课后题知 $ E_{n} $ 是一个 $ [n, n-1,2] $ 线性码, 标准生成阵为
$$
G=\left(\begin{array}{cccccc}
1 & 0 & 0 & \cdots & 0 & 1 \\
0 & 1 & 0 & \cdots & 0 & 1 \\
\vdots & \vdots & \vdots & & \vdots & \vdots \\
0 & 0 & 0 & \cdots & 1 & 1
\end{array}\right)
$$
设 $ L=\{x G \mid \forall x \in V(n-1, q)\} $, 对 $ n $ 用归纳法证明, 则 $ L $ 中的元均具有偶重量, 从而 $ L \subseteq E_{n} $.
又 $ \operatorname{dim} L=\operatorname{dim} E_{n}=n-1 $, 故 $ L=E_{n} $.

$ \left(\forall x \in L, x=v_{j 1}+v_{j 2}+\cdots+v_{j k}\right. $, 系数取在 $ F_{2} $ 中 $ \{0,1\}, k= 2$ 时, $ x=v_{j 1}+v_{j 2}, \omega(x)=\omega\left(v_{j 1}\right)+\omega\left(v_{j 2}\right)-2 \omega\left(v_{j 1} \cap v_{j 2}\right) $ 为偶数
容易用归纳法证得)
\end{solution}



\begin{exercise}
证明: 对于任意一个二元线性码 $ L $, 一定满足下列条件之一.\\
(1) L中所有码字都具有偶数重量;\\
(2) $ L $ 中一半码字具有偶数重量, 另一半码字具有奇数重量.
\end{exercise}
\begin{solution}
    证明: 设 $ L $ 是一个二元 $ [n, k] $ 线性码, 且 $ L $ 中存在码字 $ x_{0} $, 使得 $ \omega\left(x_{0}\right) $ 为奇数.
令 $ L_{1}=\{x \in L \mid \omega(x) $ 是偶数 $ \} $,
$ L_{2}=\{x \in L \mid \omega(x) $ 是奇数 $ \} $,
$ x_{0}+L_{1}=\left\{x_{0}+x \mid \forall x \in L_{1}\right\} , x_{0}+L_{1} $ 中元素重量为奇数, $ x_{0}+L_{1} \subseteq L_{2} $
同理 $ x_{0}+L_{2} \subseteq L_{1} $.
$$
\left|L_{2}\right| \geq\left|x_{0}+L_{1}\right|=\left|L_{1}\right| \geq\left|x_{0}+L_{2}\right|=\left|L_{2}\right| \Rightarrow\left|L_{1}\right|=\left|L_{2}\right| .
$$
(否定一个证另一个)

(方法二)令
$$
\begin{array}{l}
E=\{\boldsymbol{x} \in L \mid W(\boldsymbol{x}) \text { 为偶数 }\}, \\
O=\{\boldsymbol{y} \in L \mid W(\boldsymbol{y}) \text { 为奇数 }\} .
\end{array}
$$
因为零向量属于 $ E $, 所以 $ E \neq \emptyset $. 假设 $ E \neq L $, 则 $ O \neq \emptyset $. 于是存在 $ \boldsymbol{y} \in O $. 令
$$
\boldsymbol{y}+E=\{\boldsymbol{y}+\boldsymbol{x} \mid \boldsymbol{x} \in E\} .
$$
显然, $ \boldsymbol{y}+E \subseteq C $, 并且 $ |\boldsymbol{y}+E|=|E| $. 因为
$$
W(\boldsymbol{y}+\boldsymbol{x})=W(\boldsymbol{y})+W(\boldsymbol{x})-2 W(\boldsymbol{y} \cap \boldsymbol{x}),
$$
所以 $ \boldsymbol{y}+E $ 中的每个码字都具有奇数重量, 即 $ \boldsymbol{y}+E \subseteq O $. 因此, $ |E|=|\boldsymbol{y}+E| \leq|O| $.今
$$
\boldsymbol{y}+O=\{\boldsymbol{y}+u \mid u \in O\} .
$$
显然, $ \boldsymbol{y}+O \subseteq E $. 丁是, $ |O|=|\boldsymbol{y}+O| \leq|E| $. 由 $ |E| \leq|O| $ 和 $ |O| \leq|E| $ 立即可得 $ |E|=|O|=\frac{1}{2}|L| $.

\end{solution}



\begin{exercise}
    设 $ L $ 是一个二元 $ [n, k] $ 线性码, 并且对于所有码字, 任一位置的分量不全为零, $ L_{1} $ 是 $ L $ 在某一固定位置上取 0 的码字构成的集合. 证明: $ L_{1} $ 是一个二元 $ [n, k-1] $ 线性码.
\end{exercise}
\begin{solution}
    设 $ L $ 是一个二元 $ [n, k] $ 线性码.
$$
\begin{array}{l}
L_{1}=\left\{x_{1} x_{2} \cdots x_{i} \cdots x_{n} \in L \mid x_{i}=0\right\}, \\
L_{2}=\left\{x_{1} x_{2} \cdots x_{i} \cdots x_{n} \in L \mid x_{i}=1\right\} .
\end{array}
$$
$ L_{1} \cup L_{2}=L, \quad L_{1} \cap L_{2}=\emptyset $, 由条件 $ L_{2} \neq \emptyset $.
设 $ x_{0} \in L_{2} $, 当 $ x \in L_{1} $ 时, $ x_{0}+x \in L $. $ x_{0}+x $ 的第 $ i $ 个分量为 1 . $ x_{0}+x \in L_{2}, x_{0}+L_{1} \subseteq L_{2} $. 同理 $ x_{0}+L_{2} \subseteq L_{1} $.
故 $ \left|L_{1}\right|=\left|L_{2}\right|=2^{k-1} $.
故 $ L_{1} $ 是一个 $ [n, k-1] $ 线性码.

(方法二)
 设 $ C_{0} $ 是 $ C $ 中第 $ i $ 个分量为 0 的码字所构成的集合, $ C_{1} $ 是 $ C $ 中第 $ i $ 个分量为 1 的码字所构成的集合. 由前提假设可知, $ C_{0} \neq \emptyset, C_{1} \neq \emptyset $. 因为对任意 $ \boldsymbol{x}, \boldsymbol{y} \in C_{0}, \boldsymbol{x}+\boldsymbol{y} $的第 $ i $ 个分量还是 0 , 所以 $ \boldsymbol{x}+\boldsymbol{y} \in C_{0} $. 因此, $ C_{0} $ 是 $ V(n, 2) $ 的一个子空间, 即 $ C_{0} $是一个二元线性码. 设 $ u \in C_{1} $, 令
$$
\begin{array}{l}
\boldsymbol{u}+C_{0}=\left\{\boldsymbol{u}+\boldsymbol{x} \mid \boldsymbol{x} \in C_{0}\right\}, \\
\boldsymbol{u}+C_{1}=\left\{\boldsymbol{u}+\boldsymbol{v} \mid \boldsymbol{v} \in C_{1}\right\} .
\end{array}
$$
不难看出, $ \boldsymbol{u}+C_{0} \subseteq C_{1}, \boldsymbol{u}+C_{1} \subseteq C_{0} $. 于是,
$$
\begin{array}{l}
\left|C_{0}\right|=\left|\boldsymbol{u}+C_{0}\right| \leq\left|C_{1}\right|, \\
\left|C_{1}\right|=\left|\boldsymbol{u}+C_{1}\right| \leq\left|C_{0}\right| .
\end{array}
$$
由 $ \left|C_{0}\right| \leq\left|C_{1}\right| $ 和 $ \left|C_{1}\right| \leq\left|C_{0}\right| $ 立即得到
$$
\left|C_{0}\right|=\left|C_{1}\right|=\frac{1}{2}|C|=2^{k-1} .
$$
因此, $ C_{0} $ 是一个二元 $ [n, k-1] $ 线性码.
\end{solution}


\begin{exercise}
设 $ L $ 是一个二元 $ [n, k] $ 线性码, 并且其生成矩阵的列向量中不包含零向量.证明: $ L $ 中所有码字的重量之和为 $ n 2^{k-1} $.
\end{exercise}
\begin{solution}
    证明: 设 $ C_{i, 1} $ 是 $ L $ 中第 $ i $ 个分量为 1 的码字所构成的集合, $ 1 \leq i \leq n $.由上题的解题过程可知
$$
\left|C_{i, 1}\right|=\frac{1}{2}|L|=2^{k-1} .
$$
将 $ L $ 中所有码字排成一个 $ 2^{k} \times n $ 阶矩阵, 其中每一行是一个码字, 则每一列中 1 的个数为 $ 2^{k-1} $. 因此, $ L $ 中所有码字的重量之和为 $ n 2^{k-1} $.
\end{solution}



\begin{exercise}
 设二元线性码 $ L $ 的生成矩阵为
$$
G=\left(\begin{array}{lllll}
1 & 0 & 0 & 1 & 1 \\
0 & 0 & 1 & 0 & 1 \\
0 & 1 & 1 & 1 & 1
\end{array}\right)
$$
试求 $ L $ 的最小距离.
\end{exercise}
\begin{solution}
对矩阵 $ G $ 进行行初等变换
$$
\begin{array}{l}
G=\left(\begin{array}{lllll}
1 & 0 & 0 & 1 & 1 \\
0 & 0 & 1 & 0 & 1 \\
0 & 1 & 1 & 1 & 1
\end{array}\right) \rightarrow\left(\begin{array}{lllll}
1 & 0 & 0 & 1 & 1 \\
0 & 1 & 1 & 1 & 1 \\
0 & 0 & 1 & 0 & 1
\end{array}\right) \rightarrow 
\left(\begin{array}{lllll}
1 & 0 & 0 & 1 & 1 \\
0 & 1 & 0 & 1 & 0 \\
0 & 0 & 1 & 0 & 1
\end{array}\right)=G^{\prime}
\end{array}
$$
$ G^{\prime} $ 是 $ L $ 的标准型生成矩阵, 那么 $ L $ 的标准型校验矩阵是
$$
H=\left(\begin{array}{lllll}
1 & 1 & 0 & 1 & 0 \\
1 & 0 & 1 & 0 & 1
\end{array}\right) .
$$
$ H $ 的任意一列线性无关, 存在两列线性相关. 因此, $ L $ 的最小距离是2.
\end{solution}
\begin{exercise}
设三元线性码 $ L $ 的生成矩阵为
$$
G=\left(\begin{array}{llll}
1 & 0 & 1 & 1 \\
0 & 1 & 1 & 2
\end{array}\right)
$$
试求 $ L $ 的最小距离, 并证明 $ L $ 是完备码.
\end{exercise}
\begin{solution}
$ G=\left(\begin{array}{ll|ll}1 & 0 & 1 & 1 \\ 0 & 1 & 1 & 2\end{array}\right)=(I_2|A) $, 
故 $ L $ 的校验阵为 $ (-A^T|I_2)=\left(\begin{array}{llll}2 & 2 & 1 & 0 \\ 2 & 1 & 0 & 1\end{array}\right)=H $.(三元码中加法和乘法遵循模3的规则)
$ H $ 中任意两列线性无关,第1.2.4列线性相关 $ \left(2 H_{1}+H_{2}=2 H_{4}\right) $,故 $ d(L)=3 $.
由于
$$
3^{2}\left(\left(\begin{array}{l}
4 \\
0
\end{array}\right)+\left(\begin{array}{l}
4 \\
1
\end{array}\right)(3-1)\right)=3^{4}
$$
因此, $ L $ 是完备码.

(方法二)
通过生成矩阵 \(G\) 生成的三元线性码 \(L\) 包含以下码字:
\[ L = \{(0, 0, 0, 0), (0, 1, 1, 2), (0, 2, 2, 1), (1, 0, 1, 1), (1, 1, 2, 0), (1, 2, 0, 2), (2, 0, 2, 2), (2, 1, 0, 1), (2, 2, 1, 0)\} \]
这些码字是通过 \(G\) 的行的所有可能的线性组合生成的,其中的加法和乘法都是在 \(GF(3)\)(即模 \(3\) 的运算)下进行的.这个线性码 \(L\) 完全展示了由 \(G\) 定义的线性空间的结构.$d(L)=w(L)=3$.$L$是一个三元 $[4,2,3]$线性码.


\end{solution}


\begin{exercise}
 设二元线性码 $ L $ 的生成矩阵为
$$
G=\left(\begin{array}{lllll}
1 & 1 & 0 & 1 & 0 \\
0 & 1 & 0 & 1 & 0
\end{array}\right) .
$$
试求 $ L $ 的标准阵,并对信道接收端接收到的字11111和10000分别进行译码.
\end{exercise}
\begin{solution}
我们首先要从给定的生成矩阵 \(G\) 生成所有可能的码字.然后,我们将使用这些码字来确定接收到的字的最近邻码字,即通过汉明距离最小的码字来译码.对于二元情况,我们只有 \(0\) 和 \(1\) 作为系数:
\begin{itemize}
    \item 使用 \(0\) 倍的每行(不使用基向量)生成码字 \(00000\).
    \item 使用 \(1\) 倍的第一行生成码字 \(11010\).
    \item 使用 \(1\) 倍的第二行生成码字 \(01010\).
    \item 使用 \(1\) 倍的第二行生成码字 \(01010\).
\end{itemize}
因此,由 \(G\) 生成的码字集合为 \(L = \{00000, 11010, 01010, 10000\}\).
为了译码接收到的字,我们需要计算它们与 \(L\) 中所有码字的汉明距离,并找到距离最小的码字.

 译码 \(11111\)的最小汉明距离为 \(2\),因此接收到的字 \(11111\) 最接近的码字是 \(11010\).译码 \(10000\)的最小汉明距离为 \(0\),意味着接收到的字 \(10000\) 已经是码 \(L\) 中的一个码字.这表明 \(10000\) 是正确传输的,没有发生错误.



    $ L $ 的标准阵为
    \begin{center}
\begin{tabular}{llll}
00000 & $ \underline{10000} $ & 01010 & 11010 \\
01000 & 11000 & 00010 & 10010 \\
00100 & 10100 & 01110 & 11110 \\
00001 & 10001 & 01011 & 11011 \\
01100 & 11100 & 00110 & 10110 \\
01001 & 11001 & 00011 & 10011 \\
00101 & 10101 & 01111 & $ \underline{11111} $ \\
01101 & 11101 & 00111 & 10111
\end{tabular}
\end{center}
因此把 11111 译为 11010,10000 译为 10000 .
\end{solution}

\begin{exercise}
设 $ L $ 是一个二元线性码, $ x $ 是在信道接收端收到的字, $ H $ 为 $ L $ 的校验矩阵. 证明: $ H x^{T} $ 恰好是 $ x $ 发生错误位置所对应的 $ H $ 的列向量的和.
\end{exercise}
\begin{solution}
    证明: 设 $ c $ 是发送的码字, $ e=x-c, e=\left(e_{1}, e_{2}, \cdots, e_{n}\right), e $ 中不为零的位置恰好是发生错误的位置.
$$
\begin{array}{l}
H(x-c)^{T}=H x^{T}-H c^{T}=H x^{T}, \\
\text { 即 } H x^{T}=H e^{T}=\sum\limits_{i=1}^{n} e_{i} H_{i} .
\end{array}
$$
\end{solution}

\begin{exercise}
设三元线性码 $ L $ 的生成矩阵为
$$
G=\left(\begin{array}{llll}
1 & 1 & 1 & 0 \\
2 & 0 & 1 & 1
\end{array}\right) .
$$
(1) 试求 $ L $ 的标准型的生成矩阵.\\
(2) 试求 $ L $ 的标准型的校验矩阵.\\
(3) 试利用伴随式译码方法对信道接收端接收到的字2121、 1201、2222分别进行译码.
\end{exercise}
\begin{solution}
$$
\begin{aligned}
(1) G= & \left(\begin{array}{llll}
1 & 1 & 1 & 0 \\
2 & 0 & 1 & 1
\end{array}\right) \rightarrow\left(\begin{array}{llll}
1 & 1 & 1 & 0 \\
0 & 1 & 2 & 1
\end{array}\right) \rightarrow 
& \left(\begin{array}{llll}
1 & 0 & 2 & 2 \\
0 & 1 & 2 & 1
\end{array}\right)=\left(\begin{array}{ll|ll}
1 & 0 & 2 & 2 \\
0 & 1 & 2 & 1
\end{array}\right)=G^{\prime}
\end{aligned}
$$
$ G^{\prime} $ 为 $ L $ 的标准型的生成矩阵.

(2) $ L $ 的标准型的校验矩阵为
$$
H=\left(\begin{array}{llll}
1 & 1 & 1 & 0 \\
1 & 2 & 0 & 1
\end{array}\right)
$$

(3)
$C=\{0000,2011,1022,1110,0121,2102,2220,1201,0212\} .$
要对接收到的字 2121、1201、2222 进行译码,我们需要计算它们与 $ H $ 的乘积 $ H x^{T} $ ,以找到对应的综合校验向量.

$$ \begin{array}{ll}x_{1} H^{T}=22, & a_{1}=2000, a_{1} H^{T}=22, x_{1}-a_{1}=0121 \\ x_{2} H^{T}=00, & a_{2}=0000, a_{2} H^{T}=00, x_{2}-a_{2}=1201 . \\ x_{3} H^{T}=02, & a_{3}=0002, a_{3} H^{T}=02, x_{3}-a_{3}=2220 .\end{array} $$

因为 $ d(C)=W(C)=3 $, 码 $ C $ 至多可以纠正一个错误, 所以 $ V(4,3) $ 中的 9 个重量不大于 1 的向量都是陪集代表元. 由于共有 $ \frac{3^{4}}{3^{2}}=9 $ 个陪集, 所以 $ V(4,3) $ 中的 9 个重量不大于 1 的向量恰好是所有陪集的代表元. 码 $ C $ 的伴随式列表为
\begin{center}
\begin{tabular}{cc}
\hline 陪集代表元 & 伴随 \\
\hline 0000 & 00 \\
1000 & 11 \\
2000 & 22 \\
0100 & 12 \\
0200 & 21 \\
0010 & 10 \\
0020 & 20 \\
0001 & 01 \\
0002 & 02 \\
\hline
\end{tabular}
\end{center}
$ S(2121)=22,2121 $ 译码为 $ 2121-2000=0121 . S(1201)=00,1201 $ 译码为 $ 1201-0000=1201 . S(2222)=02,2222 $ 译码为 $ 2222-0002=2220 $.
\end{solution}
\begin{exercise}
  设二元线性码 $ L $ 的校验矩阵为 $ H, \widehat{L} $ 是由 $ L $ 增加一个奇偶校验位得到的扩充码.证明: $ \widehat{L} $ 是线性码, 并且其校验矩阵为
$$
\hat{H} = \left(
\begin{array}{c|c}
H & \begin{matrix} 0 \\ 0 \\ \vdots \\ 0 \end{matrix} \\
\hline
\begin{matrix} 1 & 1 & \cdots & 1 \end{matrix} & 1
\end{array}
\right).
$$

\end{exercise}
\begin{solution}
    证明: 对任意 $ x=x_{1} x_{2} \cdots x_{n} x_{n+1} \in \widehat{L}, y=y_{1} y_{2} \cdots y_{n} y_{n+1} \in $ $ \widehat{L} $, 设 $ z=x+y=z_{1} z_{2} \cdots z_{n} z_{n+1} $.
因为 $ x_{1} x_{2} \cdots x_{n}+y_{1} y_{2} \cdots y_{n} \in L $, 所以 $ z_{1} z_{2} \cdots z_{n} \in L $,而 $ \sum\limits_{i=1}^{n+1} x_{i} \equiv \sum\limits_{i=1}^{n+1} y_{i} \equiv 0(\bmod 2) $,因此有
$$
\sum_{i=1}^{n+1} z_{i}=\sum_{i=1}^{n+1}\left(x_{i}+y_{i}\right) \equiv 0(\bmod 2)
$$
从而 $ z \in \widehat{L} $, 即 $ \widehat{L} $ 是线性码.
对于任意 $ \widehat{x}=x x_{n+1} \in \widehat{L} $, 其中 $ x=x_{1} x_{2} \cdots x_{n} \in L $
$$
\overline{H} \widehat{x}^{T}=\left(\begin{array}{c}
H x^{T} \\
\sum\limits_{i=1}^{n+1} x_{i}
\end{array}\right)=\left(\begin{array}{c}
0 \\
\vdots \\
0 \\
0
\end{array}\right) .
$$
若 $ L $ 是 $ [n, k] $ 线性码, 则 $ \widehat{L} $ 是 $ [n+1, k] $ 线性码, $ \operatorname{rank}(H)=n-k $, 从而 $ \operatorname{rank}(\overline{H})=n-k+1=(n+1)-k $, 因此 $ \overline{H} $ 是 $ \widehat{L} $ 的校验矩阵.
\end{solution}


\begin{exercise}
设二元线性码 $ L $ 的生成矩阵为
$$
G = \left(
\begin{array}{c|c}
I_7 & \begin{matrix}
1 & 1 & 0 & 0 \\
1 & 0 & 1 & 0 \\
0 & 1 & 1 & 0 \\
1 & 1 & 1 & 1 \\
1 & 1 & 0 & 1 \\
0 & 1 & 0 & 1 \\
1 & 0 & 0 & 1
\end{matrix}
\end{array}
\right).
$$
试求 $ L $ 的校验矩阵, 并求其最小距离.
\end{exercise}
\begin{solution}
 $ L $ 的校验矩阵为
$$
H=\left(\begin{array}{lllllll|llll}
1 & 1 & 0 & 1 & 1 & 0 & 1 & 1 & 0 & 0 & 0 \\
1 & 0 & 1 & 1 & 1 & 1 & 0 & 0 & 1 & 0 & 0 \\
0 & 1 & 1 & 1 & 0 & 0 & 0 & 0 & 0 & 1 & 0 \\
0 & 0 & 0 & 1 & 1 & 1 & 1 & 0 & 0 & 0 & 1
\end{array}\right)
$$
$ H $ 的任意两列线性无关, 前三列线性相关(第一列是第二、三列的加和). 因此 $ L $ 的最小距离为 3 .
\end{solution}

\begin{exercise}
设 5 元线性码 $ L $ 的生成矩阵为
$$
G=\left(\begin{array}{lllll}
1 & 2 & 4 & 0 & 3 \\
0 & 2 & 1 & 4 & 1 \\
2 & 0 & 3 & 1 & 4
\end{array}\right)
$$
(1) 试求 $ L $ 的标准型的生成矩阵.\\
(2) 试求 $ L $ 的标准型的校验矩阵.\\
(3) 试求 $ L $ 的最小距离.
\end{exercise}
\begin{solution}
(1)
$$
\begin{array}{l}
 G=\left(\begin{array}{lllll}
1 & 2 & 4 & 0 & 3 \\
0 & 2 & 1 & 4 & 1 \\
2 & 0 & 3 & 1 & 4
\end{array}\right) \rightarrow\left(\begin{array}{lllll}
1 & 2 & 4 & 0 & 3 \\
0 & 2 & 1 & 4 & 1 \\
0 & 1 & 0 & 1 & 3
\end{array}\right) \rightarrow 
\left(\begin{array}{lllll}
1 & 2 & 4 & 0 & 3 \\
0 & 1 & 0 & 1 & 3 \\
0 & 2 & 1 & 4 & 1
\end{array}\right) \rightarrow\left(\begin{array}{lllll}
1 & 2 & 4 & 0 & 3 \\
0 & 1 & 0 & 1 & 3 \\
0 & 0 & 1 & 2 & 0
\end{array}\right) \rightarrow \\
\left(\begin{array}{lllll}
1 & 2 & 0 & 2 & 3 \\
0 & 1 & 0 & 1 & 3 \\
0 & 0 & 1 & 2 & 0
\end{array}\right) \rightarrow\left(\begin{array}{lllll}
1 & 0 & 0 & 0 & 2 \\
0 & 1 & 0 & 1 & 3 \\
0 & 0 & 1 & 2 & 0
\end{array}\right)=G^{\prime}
\end{array}
$$
$ G^{\prime} $ 为 $ L $ 的标准型的生成矩阵.

(2) $ L $ 的标准型的校验矩阵为
$$
H=\left(\begin{array}{lllll}
0 & 4 & 3 & 1 & 0 \\
3 & 2 & 0 & 0 & 1
\end{array}\right)
$$

(3) $ L $ 的校验矩阵 $ H $ 中存在两列线性相关(第一列是第五列的三倍,注意是5元),任意一列线性无关. 因此 $ L $ 的最小距离为 2 .
\end{solution}


\begin{exercise}
设 $ H $ 是一个 $ q $ 元 $ [n, k] $ 线性码 $ L $ 的校验矩阵, $ K $ 是 $ F_{q} $ 上的一个 $ n-k $ 阶可逆方阵. 证明: $ K H $ 仍然是 $ L $ 的校验矩阵.
\end{exercise}
\begin{solution}
 我们知道, $ L=\left\{\boldsymbol{x} \in V(n, q) \mid \boldsymbol{x} \boldsymbol{H}^{\mathrm{T}}=\mathbf{0}\right\} $. 显然, 对任意 $ \boldsymbol{x} \in L $,
$$
\boldsymbol{x}(\boldsymbol{K} \boldsymbol{H})^{\mathrm{T}}=\boldsymbol{x} \boldsymbol{H}^{\mathrm{T}} \boldsymbol{K}^{\mathrm{T}}=\mathbf{0} \boldsymbol{K}^{\mathrm{T}}=\mathbf{0} .
$$
反过来, 对任意 $ \boldsymbol{y} \in V(n, q) $, 如果 $ \boldsymbol{y}(\boldsymbol{K} \boldsymbol{H})^{\mathrm{T}}=\mathbf{0} $, 则由于 $ \boldsymbol{K} $ 可逆, 所以 $ \boldsymbol{x} \boldsymbol{H}^{\mathrm{T}}=\mathbf{0} $.于是, $ \boldsymbol{y} \in L $. 因此, $ L=\left\{\boldsymbol{x} \in V(n, q) \mid \boldsymbol{x}(\boldsymbol{K} \boldsymbol{H})^{\mathrm{T}}=\mathbf{0}\right\}, \boldsymbol{K} \boldsymbol{H} $ 还是线性码 $ L $ 的校验矩阵.

另一种证明方法: 因为 $ \boldsymbol{K} $ 可逆, 所以 $ \operatorname{Rank}(\boldsymbol{K} \boldsymbol{H})=\operatorname{Rank}(\boldsymbol{H})=n-k= $ $ \operatorname{dim}\left(L^{\perp}\right) $. 由于 $ \boldsymbol{K} \boldsymbol{H} $ 的每个行向量都是 $ \boldsymbol{H} $ 的行向量的一个线性组合,所以 $ \boldsymbol{K} \boldsymbol{H} $ 的每个行向量都是 $ L^{\perp} $ 的一个码字. 因此, $ \boldsymbol{K} \boldsymbol{H} $ 还是 $ L^{\perp} $ 的一个生成矩阵, 即 $ \boldsymbol{K} \boldsymbol{H} $ 为 $ L $ 的校验矩阵.

\end{solution}



\begin{exercise}
   对于一个 $ q $ 线性码 $ L $, 如果对任意 $ x, y \in L $, 都有 $ x \cdot y=0 $, 则称 $ L $ 是自正交的.假设 $ L $ 是一个自正交的二元线性码.证明: $ L $ 中的所有码字都具有偶数重量, 并且分量全为 1 的向量 $ 11 \cdots 1 \in L^{\perp} $.
\end{exercise}
\begin{solution}
    证明: 因为 $ L $ 是二元自正交的, 所以对 $ \forall x=x_{1} x_{2} \cdots x_{n} \in L $ 有
$$
x \cdot x=\sum_{i=1}^{n} x_{i}^{2}=\left(x_{1}+x_{2}+\cdots+x_{n}\right)^{2}=0 .
$$
因此 $ x_{1}+x_{2}+\cdots+x_{n}=0 $, 且 $ \boldsymbol{1} \cdot x=x_{1}+x_{2}+\cdots+x_{n}=0 $.所以 $ L $ 中的所有码字都具有偶数重量, 并且分量全为 1 的向量 $ 11 \cdots 1 \in L^{\perp} $.
\end{solution}



\begin{exercise}
 证明: $ q $ 元 $ [n, k] $ 线性码 $ L $ 是自对偶的充分必要条件为 $ L $ 是自正交的,并且 $ k=n / 2 $.
\end{exercise}
\begin{solution}
    证明: 必要性. 由 $ L=L ^\perp $ 可知 $ L $ 是自正交的, $ k=\operatorname{dim} L=\operatorname{dim} L^{\perp}=n-k, k=n / 2, n $ 为偶数.

充分性. 由 $ L $ 是自正交的可知 $ L \subseteq L^{\perp} $, 再由 $ k=n / 2=\operatorname{dim} L $ 可得 $ \operatorname{dim} L^{\perp}=n-k=n / 2 $, 因此 $ L=L^{\perp} $.
\end{solution}




\begin{exercise}
设 $ L $ 是一个二元 $ [n, k] $ 线性码,并且分量全为 1 的向量 $ 11 \cdots 1 \in L $. 证明:对于 $ i=0,1, \cdots, n $, 我们有
$$
A_{i}=A_{n-i}
$$
其中 $ A_{i} $ 表示 $ L $ 中重量为 $ i $ 的码字的数目.
\end{exercise}
\begin{solution}
    证明: 令 $ L_{i}=\{\boldsymbol{x} \in L \mid \omega(x)=i\} $, 由于当 $ \boldsymbol{x }\in L $ 时, $ \boldsymbol{x}+11 \cdots 1 \in L $, 所以 $ \boldsymbol{x} \longmapsto x+11 \cdots 1 $ 是 $ L_{i} $ 到 $ L_{n-i} $ 间的一个一一对应. 于是$|L_{i}|=|L_{n-i}|$,从而 $ A_{i}=A_{n-i} $.
\end{solution}



\begin{exercise}
 设二元线性码 $ L $ 的生成矩阵为
$$
G=\left(\begin{array}{lllll}
1 & 0 & 0 & 1 & 1 \\
0 & 1 & 0 & 0 & 1 \\
0 & 0 & 1 & 0 & 1
\end{array}\right) .
$$
试求 $ L $ 的重量分布.
\end{exercise}
\begin{solution}
$ L $ 的校验矩阵为
$$
H=\left(\begin{array}{lllll}
1 & 0 & 0 & 1 & 0 \\
1 & 1 & 1 & 0 & 1
\end{array}\right) .
$$
$ L $ 的对偶码 $ L^{\perp}=\{00000,10010,11101,01111\} $
$$
W_{L^{\perp}}(z)=1+z^{2}+2 z^{4}
$$
由二元线性码的Mac Williams恒等式知
$$
\begin{aligned}
W_{L}(z) & =\frac{1}{2^{2}}(1+z)^{5} W_{L^{\perp}}\left(\frac{1-z}{1+z}\right) \\
& =\frac{1}{4}(1+z)^{5}\left[1+\left(\frac{1-z}{1+z}\right)^{2}+2\left(\frac{1-z}{1+z}\right)^{4}\right] \\
& =1+3 z^{2}+3 z^{3}+z^{5} .
\end{aligned}
$$
因此线性码 $ L $ 的重量分布为
$$
A_{0}=1, A_{1}=0, A_{2}=3, A_{3}=3, A_{4}=0, A_{5}=1 .
$$
\end{solution}


\begin{exercise}
 设二元 $ [9,7] $ 线性码 $ L $ 的生成矩阵为
$$
G = \left[
\begin{array}{c|cc}
I_7 & \begin{matrix}
0 & 1 \\
0 & 1 \\
1 & 0 \\
1 & 0 \\
1 & 1 \\
1 & 1 \\
1 & 1
\end{matrix}
\end{array}
\right].
$$
试利用二元线性码的Mac Williams恒等式确定 $ A_{0}, A_{1}, A_{2}, A_{3} $ 的值, 并证明分量全为 1 的向量 $ 11 \cdots 1 \in L $. 进一步, 确定 $ L $ 的重量分布多项式
$$
W_{L}(z)=\sum_{i=0}^{9} A_{i} z^{i}
$$
\end{exercise}
\begin{solution}
 $ L $ 的校验矩阵为
$$
H=\left(\begin{array}{lllllll|ll}
0 & 0 & 1 & 1 & 1 & 1 & 1 & 1 & 0 \\
1 & 1 & 0 & 0 & 1 & 1 & 1 & 0 & 1
\end{array}\right) .
$$
$ L $ 的对偶码$L^{\perp}=\{000000000,001111110,110011101,111100011\}$,
$$
W_{L^{\perp}}(z)=1+3 z^{6} .
$$
由二元线性码的Mac Williams恒等式知
$$
\begin{aligned}
W_{L}(z)&=\frac{1}{2^{2}}(1+z)^{9} W_{L^{\perp}}\left(\frac{1-z}{1+z}\right) \\
&=\frac{1}{2^{2}}(1+z)^{9}\left[1+3\left(\frac{1-z}{1+z}\right)^{6}\right]\\
&=\frac{1}{2^{2}}\left[(1+z)^{9}+3(1-z)^{6}(1+z)^{3}\right] \\
&=1+9 z^{2}+27 z^{3}+27 z^{4}+27 z^{5}+27 z^{6}+9 z^{7}+z^{9} .
\end{aligned}
$$
由此可知 $ A_{0}=1, A_{1}=0, A_{2}=9, A_{3}=27 $, 再由 $ A_{9}=1 $ 知向量 $ 11 \cdots 1 \in L $.
\end{solution}


\begin{exercise}
设 $ L $ 是一个二元线性码, $ L_{0} $ 是 $ L $ 中所有具有偶数重量的码字构成的码.证明:
$$
W_{L_{0}}(z)=\frac{1}{2}\left(W_{L}(z)+W_{L}(-z)\right) .
$$
\end{exercise}
\begin{solution}
    证明:
$$
\begin{aligned}
&\frac{1}{2}\left(W_{L}(z)+W_{L}(-z)\right)\\
&=\frac{1}{2}\left[\left(A_{0}+A_{1} z+\cdots+A_{n} z^{n}\right)+\left(A_{0}+A_{1}(-) z+\cdots+A_{n}(-z)^{n}\right)\right] \\
&=A_{0}+A_{2} z^{2}+\cdots+A_{\left\lfloor\frac{n}{2}\right\rfloor} z^{\left\lfloor\frac{n}{2}\right\rfloor}=W_{L_{0}}(z) .
\end{aligned}
$$
\end{solution}

\begin{exercise}
    设 $ L $ 是一个二元线性码, $ \widehat{L} $ 是由 $ L $ 增加一个奇偶校验位构成的扩充码.证明:
$$
W_{\hat{L}}(z)=\frac{1}{2}\left[(1+z) W_{L}(z)+(1-z) W_{L}(-z)\right] .
$$
\end{exercise}
\begin{solution}
证明: 设 $ W_{L}(z)=\sum\limits_{i=0}^{n} A_{i} z^{i} $, 则
$$
\begin{aligned}
W_{\widehat{L}}(z)&=\sum_{0 \leq i \leq n+1, i \text { 是偶数 }}\left(A_{i}+A_{i-1}\right) z^{i} \\
&=\sum_{0 \leq i \leq n+1, i \text { 是偶数 }} A_{i} z^{i}+\sum_{0 \leq i \leq n, i \text { 是奇数 }} A_{j} z^{j+1} \\
&=\frac{1}{2}\left(W_{L}(z)+W_{L}(-z)\right)+\frac{1}{2} z\left(W_{L}(z)-W_{L}(-z)\right) \\
&=\frac{1}{2}\left[(1+z) W_{L}(z)+(1-z) W_{L}(-z)\right] .
\end{aligned}
$$
\end{solution}